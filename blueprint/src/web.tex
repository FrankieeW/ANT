\documentclass[11pt]{article}
\usepackage{blueprint}

\title{ANT: Algebraic Number Theory in Lean}
\author{Formalization Project}

\begin{document}

\maketitle

\section{Definitions}

\begindefinition{R}{def:R}
  The quadratic integer ring $\mathbb{Z}[\sqrt{-5}]$ used for ideal factorization examples.
  \lean{../ANT/Basic.lean}
  \uses{}
\enddefinition

\begindefinition{isPrime}{def:isPrime}
  An ideal is prime if it contains only zero divisors and whenever $ab \in I$, either $a \in I$ or $b \in I$.
  \lean{../ANT/Ideals.lean}
\enddefinition

\section{Main Theorems}

\begintheorem{factorization\_of\_two}{thm:factorization_of_two}
  $\langle 2 \rangle = \langle 2, 1 + \sqrt{-5} \rangle^2$ in $\mathbb{Z}[\sqrt{-5}]$.
  \lean{../ANT/Ideals.lean}
  \leanok
  \uses{def:R}
\endtheorem

\begintheorem{factorization\_of\_three}{thm:factorization_of_three}
  $\langle 3 \rangle = \langle 3, 1 + \sqrt{-5} \rangle \cdot \langle 3, 1 - \sqrt{-5} \rangle$ in $\mathbb{Z}[\sqrt{-5}]$.
  \lean{../ANT/Ideals.lean}
  \leanok
  \uses{def:R}
\endtheorem

\begintheorem{factorization\_of\_one\_plus\_sqrtd}{thm:factorization_one_plus}
  $\langle 1 + \sqrt{-5} \rangle = \langle 2, 1 + \sqrt{-5} \rangle \cdot \langle 3, 1 + \sqrt{-5} \rangle$ in $\mathbb{Z}[\sqrt{-5}]$.
  \lean{../ANT/Ideals.lean}
  \leanok
  \uses{def:R, thm:factorization_of_two, thm:factorization_of_three}
\endtheorem

\begintheorem{factorization\_of\_one\_minus\_sqrtd}{thm:factorization_one_minus}
  $\langle 1 - \sqrt{-5} \rangle = \langle 2, 1 - \sqrt{-5} \rangle \cdot \langle 3, 1 - \sqrt{-5} \rangle$ in $\mathbb{Z}[\sqrt{-5}]$.
  \lean{../ANT/Ideals.lean}
  \leanok
  \uses{def:R, thm:factorization_of_two, thm:factorization_of_three}
\endtheorem

\begintheorem{ideal\_of\_prime\_norm\_is\_prime}{thm:prime_norm}
  An ideal whose absolute norm is a prime number is a prime ideal.
  \lean{../ANT/Ideals.lean}
  \leanok
\endtheorem

\begintheorem{isPrime\_span\_two\_one\_plus\_sqrtd}{thm:prime_2_1p}
  $\langle 2, 1 + \sqrt{-5} \rangle$ is a prime ideal in $\mathbb{Z}[\sqrt{-5}]$.
  \lean{../ANT/Ideals.lean}
  \leanok
  \uses{def:R}
\endtheorem

\begintheorem{isPrime\_span\_three\_one\_plus\_sqrtd}{thm:prime_3_1p}
  $\langle 3, 1 + \sqrt{-5} \rangle$ is a prime ideal in $\mathbb{Z}[\sqrt{-5}]$.
  \lean{../ANT/Ideals.lean}
  \leanok
  \uses{def:R}
\endtheorem

\begintheorem{isPrime\_span\_three\_one\_minus\_sqrtd}{thm:prime_3_1m}
  $\langle 3, 1 - \sqrt{-5} \rangle$ is a prime ideal in $\mathbb{Z}[\sqrt{-5}]$.
  \lean{../ANT/Ideals.lean}
  \leanok
  \uses{def:R}
\endtheorem

\section{Lemmas}

\beginlemma{principal\_eq\_of\_le\_of\_le}{lem:principal_eq}
  If $I \le J$ and $J \le I$ then $I = J$.
  \lean{../ANT/Ideals.lean}
  \leanok
\endlemma

\beginlemma{in\_span\_of\_eq}{lem:in_span}
  If $x = y$ and $y \in I$, then $x \in I$.
  \lean{../ANT/Ideals.lean}
  \leanok
\endlemma

\beginlemma{span\_le\_span\_singleton\_of\_forall\_dvd}{lem:span_le}
  If $a$ divides every element of $S$, then $\langle S \rangle \le \langle a \rangle$.
  \lean{../ANT/Ideals.lean}
  \leanok
\endlemma

\beginlemma{mem\_span\_two\_one\_plus\_sqrtd\_iff}{lem:mem_2_1p}
  $z \in \langle 2, 1 + \sqrt{-5} \rangle$ iff $z.re + z.im$ is even.
  \lean{../ANT/Ideals.lean}
  \leanok
  \uses{def:R}
\endlemma

\beginlemma{mem\_span\_three\_one\_plus\_iff}{lem:mem_3\_sqrtd_1p}
  $z \in \langle 3, 1 + \sqrt{-5} \rangle$ iff $3 \mid (z.re - z.im)$.
  \lean{../ANT/Ideals.lean}
  \leanok
  \uses{def:R}
\endlemma

\beginlemma{mem\_span\_three\_one\_minus\_sqrtd\_iff}{lem:mem_3_1m}
  $z \in \langle 3, 1 - \sqrt{-5} \rangle$ iff $3 \mid (z.re + z.im)$.
  \lean{../ANT/Ideals.lean}
  \leanok
  \uses{def:R}
\endlemma

\nocite{*
}

\bibliographystyle{plain}
\bibliography{references}

\end{document}
